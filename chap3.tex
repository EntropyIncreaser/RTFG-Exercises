%!TEX TS-program = xelatex
%!TEX encoding = UTF-8 Unicode
\documentclass[11pt]{report}
\usepackage[margin = 1in]{geometry}

\usepackage{amsmath}
\usepackage{amssymb}
\usepackage{mleftright}
\usepackage{enumitem}
\usepackage{textcomp, gensymb}
\usepackage{tikz, tikz-cd}
\usepackage{bbding}
\usepackage{tabularx}
\usepackage{nicematrix}
\usepackage{extarrows}
\usepackage{lastpage}
\usepackage{hyperref}

\usepackage{fancyhdr}  % Header and Footer formatting

\usetikzlibrary{decorations.markings}

% \usepackage{mathpazo}
\usepackage{unicode-math}
\setmainfont[
	BoldFont={TeX Gyre Termes Bold},
	ItalicFont={TeX Gyre Termes Italic},
	BoldItalicFont={TeX Gyre Termes Bold Italic},
	PunctuationSpace=2
]{TeX Gyre Termes}
\setmathfont
	[Extension = .otf,
	math-style= TeX,
  BoldFont = XITSMath-Bold.otf,
  BoldItalicFont = XITS-BoldItalic.otf
]{XITSMath-Regular}

\usepackage{multicol}

\hypersetup{colorlinks = true,
            linkcolor = blue,
            citecolor = red,
            urlcolor = teal}

\pagestyle{fancy}
\renewcommand{\headrulewidth}{0.4pt}
\renewcommand{\footrulewidth}{0.4pt}
\setlength{\headheight}{18pt}

% Header and Footer Information
\lhead{\small\emph{Baitian Li}}
\chead{}
\rhead{\textsc{Representation Theory of Finite Groups}}
\lfoot{\today}
\cfoot{}
\rfoot{\thepage\ of \pageref{LastPage}}  % Counts the pages.

\makeatletter % This provides a total page count as \ref{NumPages}
\AtEndDocument{\immediate\write\@auxout{\string\newlabel{NumPages}{{\thepage}}}}
\makeatother

%\lineskiplimit=-\maxdimen\relax

\usepackage{amsthm}  % This will create the Problem environment
\newtheorem*{lemma}{Lemma}
\newtheorem*{theorem}{Theorem}
\newtheoremstyle{mythm}%
    {12pt}{}%
    {\sffamily}{}%
    {\bfseries \sffamily}{.}%
    {.5em}%
    {\thmname{#1}\thmnumber{ #2}\thmnote{ #3}}

\theoremstyle{mythm}

\expandafter\let\expandafter\oldproof\csname\string\proof\endcsname
\let\oldendproof\endproof
\renewenvironment{proof}[1][\proofname]{%
  \oldproof[\normalfont \bfseries #1]%
}{\oldendproof}

\newtheorem{exercise}{Exercise}[chapter]
\renewcommand*{\proofname}{Proof}

\newtheoremstyle{myans}%
    {12pt}{}%
    {}{}%
    {\bfseries}{.}%
    {.5em}%
    {\thmname{#1}\thmnumber{ #2}\thmnote{ #3}}

\theoremstyle{myans}
\newtheorem*{answer}{Answer}

\setlist[enumerate]{noitemsep, topsep = 0.2em}
\setlist[enumerate, 1]{label = {\arabic*.}}
% \setlist[enumerate, 2]{label = {(\alph*)}}
\setlist[description]{topsep = 0.2em, listparindent = \parindent, font = \normalfont,
  itemsep = 0em}

\newcommand{\bbR}{\mathbb R}
\newcommand{\bbN}{\mathbb N}
\newcommand{\bbZ}{\mathbb Z}
\newcommand{\bbQ}{\mathbb Q}
\newcommand{\bbC}{\mathbb C}
\newcommand{\Id}{\mathit{Id}}

\DeclareMathOperator{\Hom}{Hom}
\DeclareMathOperator{\End}{End}
\DeclareMathOperator{\Tr}{Tr}
\DeclareMathOperator{\sgn}{sgn}
\DeclareMathOperator{\rank}{rank}
\DeclareMathOperator{\GL}{GL}

\begin{document}

\setcounter{chapter}{2}
\chapter{Group Representations}

\begin{exercise}
  Let $\varphi\colon D_4 \to \GL_2(\bbC)$ be the representation given by
  \[ \varphi(r^k) = \begin{bmatrix}
    i^k & 0 \\ 0 & (-i)^k
  \end{bmatrix}, \varphi(sr^k) = \begin{bmatrix}
    0 & (-i)^k \\ i^k & 0
  \end{bmatrix} \]
  where $r$ is rotation counterclockwise by $\pi/2$ and $s$ is reflection over the $x$-axis.
  Prove that $\varphi$ is irreducible.
  \begin{proof}
    The eigenvectors of $\varphi(r)$ are $\alpha e_1$ and $\beta e_2$ for $\alpha, \beta\neq 0$.
    But clearly they are not the eigenvector of $\varphi(sr)$. Since $\varphi$ has degree $2$
    and has no common eigenvector, $\varphi$ is irreducible.
  \end{proof}
\end{exercise}

\setcounter{exercise}{4}
\begin{exercise}
  Let $\varphi\colon G \to \GL(V )$ be a representation of a finite group $G$. Define
  the \emph{fixed subspace}
  \[ V^G = \{ v\in V \mid \varphi_g v = v, \forall g\in G \}. \]
  \begin{enumerate}
    \item Show that $V^G$ is a $G$-invariant subspace.
    \begin{proof}
      It's clear that $V^G$ forms a subspace. For each $v\in V^G$, since $\varphi_g v = v$,
      we have $\varphi_g v \in V^G$, since this holds for all $g$, $V^G$ is $G$-invariant.
    \end{proof}
    \item Show that
    \[ \frac 1{|G|} \sum_{h\in G} \varphi_h v \in V^G \]
    for all $v\in V$.
    \begin{proof}
      Let such vector be $w$.
      For each $g\in G$, we have
      \[ \varphi_g w = \frac 1{|G|} \sum_{h\in G} \varphi_g \varphi_h v
      = \frac 1{|G|} \sum_{k\in G} \varphi_k v = w, \]
      so $w\in V^G$.
    \end{proof}
    \item Show that if $v \in V^G$, then
    \[ \frac 1{|G|} \sum_{h\in G} \varphi_h v = v. \]
    \begin{proof}
      By the definition of $V^G$,
      \[ \frac 1{|G|} \sum_{h\in G} \varphi_h v = \frac 1{|G|} \sum_{h\in G} v = v. \qedhere\]
    \end{proof}
    \item Conclude $\dim V^G$ is the rank of the operator
    \[ P = \frac 1{|G|} \sum_h \varphi_h. \]
    \begin{proof}
      By the preceding argument, the image of $P$ is exacly $V^G$, thus $\dim V^G = \rank P$.
    \end{proof}
    \item Show that $P^2 = P$.
    \begin{proof}
      For each $v$, since $Pv \in V^G$, we have $P(Pv) = Pv$, thus $P^2 = P$.
    \end{proof}
    \item Conclude $\Tr(P)$ is the rank of $P$.
    \begin{proof}
      Since $P^2 - P = 0$, $P$ can be diagonalized and contains only eigenvalues $0$ and $1$.
      So $\rank P$ is the sum of eigenvalues with multiplicities, which is $\Tr (P)$.
    \end{proof}
    \item Conclude
    \[ \dim V^G = \frac 1{|G|} \sum_{h\in G} \Tr(\varphi_h). \]
    \begin{proof}
      $\dim V^G = \rank P = \Tr (P) = \frac 1{|G|} \sum_{h\in G} \Tr (\varphi_h)$.
    \end{proof}
  \end{enumerate}
\end{exercise}

\end{document}
