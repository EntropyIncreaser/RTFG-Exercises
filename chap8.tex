%!TEX TS-program = xelatex
%!TEX encoding = UTF-8 Unicode
\documentclass[11pt]{report}
\usepackage[margin = 1in]{geometry}

\usepackage{amsmath}
\usepackage{amssymb}
\usepackage{mleftright}
\usepackage{enumitem}
\usepackage{textcomp, gensymb}
\usepackage{tikz, tikz-cd}
\usepackage{bbding}
\usepackage{tabularx}
\usepackage{booktabs}
\usepackage{nicematrix}
\usepackage{extarrows}
\usepackage{lastpage}
\usepackage{hyperref}

\usepackage{fancyhdr}  % Header and Footer formatting

\usetikzlibrary{decorations.markings}

% \usepackage{mathpazo}
\usepackage{unicode-math}
\setmainfont[
	BoldFont={TeX Gyre Termes Bold},
	ItalicFont={TeX Gyre Termes Italic},
	BoldItalicFont={TeX Gyre Termes Bold Italic},
	PunctuationSpace=2
]{TeX Gyre Termes}
\setmathfont
	[Extension = .otf,
	math-style= TeX,
  BoldFont = XITSMath-Bold.otf,
  BoldItalicFont = XITS-BoldItalic.otf
]{XITSMath-Regular}

\usepackage{multicol}

\hypersetup{colorlinks = true,
            linkcolor = blue,
            citecolor = red,
            urlcolor = teal}

\pagestyle{fancy}
\renewcommand{\headrulewidth}{0.4pt}
\renewcommand{\footrulewidth}{0.4pt}
\setlength{\headheight}{18pt}

% Header and Footer Information
\lhead{\small\emph{Baitian Li}}
\chead{}
\rhead{\textsc{Representation Theory of Finite Groups}}
\lfoot{\today}
\cfoot{}
\rfoot{\thepage\ of \pageref{LastPage}}  % Counts the pages.

\makeatletter % This provides a total page count as \ref{NumPages}
\AtEndDocument{\immediate\write\@auxout{\string\newlabel{NumPages}{{\thepage}}}}
\makeatother

%\lineskiplimit=-\maxdimen\relax

\usepackage{amsthm}  % This will create the Problem environment
\newtheorem*{lemma}{Lemma}
\newtheorem*{theorem}{Theorem}
\newtheoremstyle{mythm}%
    {12pt}{}%
    {\sffamily}{}%
    {\bfseries \sffamily}{.}%
    {.5em}%
    {\thmname{#1}\thmnumber{ #2}\thmnote{ #3}}

\theoremstyle{mythm}

\expandafter\let\expandafter\oldproof\csname\string\proof\endcsname
\let\oldendproof\endproof
\renewenvironment{proof}[1][\proofname]{%
  \oldproof[\normalfont \bfseries #1]%
}{\oldendproof}

\newtheorem{exercise}{Exercise}[chapter]
\renewcommand*{\proofname}{Proof}

\newtheoremstyle{myans}%
    {12pt}{}%
    {}{}%
    {\bfseries}{.}%
    {.5em}%
    {\thmname{#1}\thmnumber{ #2}\thmnote{ #3}}

\theoremstyle{myans}
\newtheorem*{answer}{Answer}

\setlist[enumerate]{noitemsep, topsep = 0.2em}
\setlist[enumerate, 1]{label = {\arabic*.}}
% \setlist[enumerate, 2]{label = {(\alph*)}}
\setlist[description]{topsep = 0.2em, listparindent = \parindent, font = \normalfont,
  itemsep = 0em}

\newcommand{\bbR}{\mathbb R}
\newcommand{\bbN}{\mathbb N}
\newcommand{\bbZ}{\mathbb Z}
\newcommand{\bbQ}{\mathbb Q}
\newcommand{\bbC}{\mathbb C}
\newcommand{\Id}{\mathit{Id}}

\DeclareMathOperator{\Hom}{Hom}
\DeclareMathOperator{\End}{End}
\DeclareMathOperator{\Tr}{Tr}
\DeclareMathOperator{\sgn}{sgn}
\DeclareMathOperator{\rank}{rank}
\DeclareMathOperator{\GL}{GL}
\DeclareMathOperator{\Ind}{Ind}
\DeclareMathOperator{\Res}{Res}

\newcommand{\ang}[1]{\langle #1 \rangle}
\newcommand{\Ang}[1]{\left\langle #1 \right\rangle}
\newcommand{\norm}[1]{\| #1 \|}

\begin{document}

\setcounter{chapter}{7}
\chapter{Induced Representations}

\setcounter{exercise}{1}
\begin{exercise}
  Let $H, K$ be subgroups of $G$. Let $\sigma\colon H \to S_{G/K}$ be given by
  $\sigma_h(gK) = hgK$. Show that the set of double cosets $H\backslash G/K$ is the set of orbits of
  $H$ on $G/K$.
  \begin{proof}
    For $xK \sim yK$, we have $h\in H$ such that $hxK = yK$, thus $HxK = HyK$. Conversely, if
    $HxK = HyK$, we have $x \in HyK$, thus $x = h y k$, thus $h^{-1}xK = yK$.
  \end{proof}
\end{exercise}

\begin{exercise}
  Let $G$ be a group with a cyclic normal subgroup $H = \ang{a}$ of order
  $k$. Suppose that $N_G(a) = H$, that is, $sa = as$ implies $s \in H$. Show that if
  $\chi\colon H \to \bbC^*$ is the character given by $\chi(a^m) = e^{2\pi im/k}$, then $\Ind^G_H \chi$ is an
  irreducible character of $G$.
  \begin{proof}
    Clearly $\chi$ itself is irreducible. For each $s\notin H$, since $s^{-1}Hs = H$, we have
    $s^{-1}as = a^r$ for some $2\leq r \leq k-1$. Therefore we have $s^{-1} a^m s = a^{mr}$.
    Therefore, $\chi^s(a^m) = \chi(a^{mr}) = e^{2\pi i mr / k}$. We have $\chi \neq \chi^s$.
    By the Mackey's criterion, $\Ind_H^G \chi$ is an irreducible character.
  \end{proof}
\end{exercise}

\setcounter{exercise}{5}
\begin{exercise}
  Let $N$ be a normal subgroup of a group $G$ and suppose that $\varphi\colon N \to \GL_d(\bbC)$ is
  a representation. For $s \in G$, define $\varphi_s \colon N \to \GL_d(\bbC)$ by
  $\varphi^s(n) = \varphi(s^{-1}ns)$. Prove that $\varphi$ is irreducible if and only if $\varphi^s$
  is irreducible.
  \begin{proof}
    For each invariant subspace $V\leq \bbC^d$ such that $\varphi_g(V) = V$ for all $g$,
    we have $\varphi^s_n(V) = \varphi_{s^{-1}ns}(V) = V$, so $V$ is also an invariant subspace
    of $\varphi^s$. Conversely, an invariant subspace of $\varphi^s$ is an invariant subspace
    of $(\varphi^s)^{s^{-1}} = \varphi$. Thus $\varphi$ is irreducible iff $\varphi^s$ is irreducible.
  \end{proof}
\end{exercise}

\begin{exercise}
  Show that if $G$ is a non-abelian group and $\varphi\colon Z(G) \to \GL_d(\bbC)$ is
  an irreducible representation of the center of $G$, then $\Ind^G_{Z(G)}\varphi$ is not irreducible.
  \begin{proof}
    Take $s\notin Z(G)$, by $g\in Z(G)$ we have $\varphi^s(g) = \varphi(s^{-1}gs) = \varphi(g)$,
    thus $\varphi^s = \varphi$. Since $G$ is non-abelian, such $s$ exists. Noticed that
    $Z(G) \trianglelefteq G$, by Mackey's criterion, $\Ind_{Z(G)}^G \varphi$ is not irreducible.
  \end{proof}
\end{exercise}

\begin{exercise}
  A representation $\varphi \colon G\to \GL(V)$ is called \emph{faithful} if it is one-to-one.
  \begin{enumerate}
    \item Let $H$ be a subgroup of $G$ and suppose $\varphi\colon H \to \GL_d(\bbC)$ is a faithful
    representation. Show that $\varphi^G = \Ind^G_H \varphi$ is a faithful representation of $G$.
    \begin{proof}
      It's clear that $\varphi$ is faithful iff $\ker \varphi = 1$. Suppose $\varphi^G(g) = I$,
      we have $[\varphi^G(g)]_{ii} = \varphi(t_i gt_i^{-1}) = I$, thus $t_igt_i^{-1} = 1$, we have
      $g = 1$.
    \end{proof}
    \item Show that every representation of a simple group which is not a direct sum of
    copies of the trivial representation is faithful.
    \begin{proof}
      Every representation can be decomposed to the direct sum of several irreducibles, it's not hard
      to see that we only need to prove that, every nontrivial irreducible representation is faithful.
      If its degree is greater than $1$, then such representation is not trivial. Otherwise consider
      $\ker \varphi$. Since $G$ is simple and $\ker \varphi \trianglelefteq G$, we can only have
      $\ker \varphi = 1$, thus $\varphi$ is faithful.
    \end{proof}
  \end{enumerate}
\end{exercise}

\begin{exercise}
  Let $G$ be a group and let $H$ be a subgroup. Let $\sigma\colon G \to S_{G/H}$ be the group action
  given by $\sigma_g(xH) = gxH$.
  \begin{enumerate}
    \item Show that $\sigma$ is transitive.
    \begin{proof}
      For each $xH$ and $yH$, we have $yH = \sigma_{yx^{-1}}(xH)$.
    \end{proof}
    \item Show that $H$ is the stabilizer of the coset $H$.
    \begin{proof}
      Trivial.
    \end{proof}
    \item Recall that if $1_H$ is the trivial character of $H$, then $\Ind^G_H 1_H$ is the character
    $\chi_{\tilde \sigma}$ of the permutation representation $\tilde \sigma: G \to \GL(\bbC(G/H))$. Use Frobenius
    reciprocity to show that the rank of $\sigma$ is the number of orbits of $H$ on $G/H$.
    \begin{proof}
      \begin{align*}
        \Ind^G_H 1_H (g) &= \frac 1{|H|} |\{x\in G\mid x^{-1}gx\in H\}|\\
        &= \frac 1{|H|} |\{x\in G\mid gxH= xH\}|\\
        &= |\{xH \in G/H \mid gxH= xH\}|\\
        &= \chi_{\tilde \sigma}(g).
      \end{align*}
      Thus we have
      \begin{align*}
        \rank \sigma &= \ang{\chi_{\tilde \sigma}, \chi_{\tilde \sigma}}\\
        &= \ang{\Ind^G_H 1_H, \chi_{\tilde \sigma}}\\
        &= \ang{1_H, \Res^G_H \chi_{\tilde \sigma}}\\
        &= \frac 1{|H|} \sum_{h\in H} \chi_{\tilde \sigma}(h).
      \end{align*}
      By Burnside's lemma, $\frac 1{|H|} \sum_{h\in H} \chi_{\tilde \sigma}(h)$ is the number of orbits of $H$ on $G/H$.
    \end{proof}
    \item Conclude that $G$ is $2$-transitive on $G/H$ if and only if $H$ is transitive on the set
    of cosets not equal to $H$ in $G/H$.
    \begin{proof}
      $G$ is $2$-transitive iff $\rank \sigma = 1$ iff the number of orbits of $H$ on $G/H$ is $1$.
    \end{proof}
    \item Show that the rank of $\sigma$ is also the number of double cosets in $H\backslash G/H$ either
    directly or by using Mackey's Theorem.
    \begin{proof}
      Directly follows from Exercise 8.2.
    \end{proof}
  \end{enumerate}
\end{exercise}

\begin{exercise}
  Use Frobenius reciprocity to give another proof that if $\rho$ is an irreducible representation
  of $G$, then the multiplicity of $\rho$ as a constituent of the regular representation is the
  degree of $\rho$.
  \begin{proof}
    Noticed that the character of regular representation $\chi = \Ind_{\{1\}}^{G} 1$,
    let the character of $\rho$ be $\chi'$, we have
    \[ \ang{\chi, \chi'} = \ang{\Ind_{\{1\}}^G 1, \chi'}
    = \ang{1, \Res_{\{1\}}^G \chi'}
    = \chi'(1) = d. \qedhere \]
  \end{proof}
\end{exercise}

\end{document}
