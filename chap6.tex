%!TEX TS-program = xelatex
%!TEX encoding = UTF-8 Unicode
\documentclass[11pt]{report}
\usepackage[margin = 1in]{geometry}

\usepackage{amsmath}
\usepackage{amssymb}
\usepackage{mleftright}
\usepackage{enumitem}
\usepackage{textcomp, gensymb}
\usepackage{tikz, tikz-cd}
\usepackage{bbding}
\usepackage{tabularx}
\usepackage{nicematrix}
\usepackage{extarrows}
\usepackage{lastpage}
\usepackage{hyperref}

\usepackage{fancyhdr}  % Header and Footer formatting

\usetikzlibrary{decorations.markings}

% \usepackage{mathpazo}
\usepackage{unicode-math}
\setmainfont[
	BoldFont={TeX Gyre Termes Bold},
	ItalicFont={TeX Gyre Termes Italic},
	BoldItalicFont={TeX Gyre Termes Bold Italic},
	PunctuationSpace=2
]{TeX Gyre Termes}
\setmathfont
	[Extension = .otf,
	math-style= TeX,
  BoldFont = XITSMath-Bold.otf,
  BoldItalicFont = XITS-BoldItalic.otf
]{XITSMath-Regular}

\usepackage{multicol}

\hypersetup{colorlinks = true,
            linkcolor = blue,
            citecolor = red,
            urlcolor = teal}

\pagestyle{fancy}
\renewcommand{\headrulewidth}{0.4pt}
\renewcommand{\footrulewidth}{0.4pt}
\setlength{\headheight}{18pt}

% Header and Footer Information
\lhead{\small\emph{Baitian Li}}
\chead{}
\rhead{\textsc{Representation Theory of Finite Groups}}
\lfoot{\today}
\cfoot{}
\rfoot{\thepage\ of \pageref{LastPage}}  % Counts the pages.

\makeatletter % This provides a total page count as \ref{NumPages}
\AtEndDocument{\immediate\write\@auxout{\string\newlabel{NumPages}{{\thepage}}}}
\makeatother

%\lineskiplimit=-\maxdimen\relax

\usepackage{amsthm}  % This will create the Problem environment
\newtheorem*{lemma}{Lemma}
\newtheorem*{theorem}{Theorem}
\newtheoremstyle{mythm}%
    {12pt}{}%
    {\sffamily}{}%
    {\bfseries \sffamily}{.}%
    {.5em}%
    {\thmname{#1}\thmnumber{ #2}\thmnote{ #3}}

\theoremstyle{mythm}

\expandafter\let\expandafter\oldproof\csname\string\proof\endcsname
\let\oldendproof\endproof
\renewenvironment{proof}[1][\proofname]{%
  \oldproof[\normalfont \bfseries #1]%
}{\oldendproof}

\newtheorem{exercise}{Exercise}[chapter]
\renewcommand*{\proofname}{Proof}

\newtheoremstyle{myans}%
    {12pt}{}%
    {}{}%
    {\bfseries}{.}%
    {.5em}%
    {\thmname{#1}\thmnumber{ #2}\thmnote{ #3}}

\theoremstyle{myans}
\newtheorem*{answer}{Answer}

\setlist[enumerate]{noitemsep, topsep = 0.2em}
\setlist[enumerate, 1]{label = {\arabic*.}}
% \setlist[enumerate, 2]{label = {(\alph*)}}
\setlist[description]{topsep = 0.2em, listparindent = \parindent, font = \normalfont,
  itemsep = 0em}

\newcommand{\bbR}{\mathbb R}
\newcommand{\bbN}{\mathbb N}
\newcommand{\bbZ}{\mathbb Z}
\newcommand{\bbQ}{\mathbb Q}
\newcommand{\bbC}{\mathbb C}
\newcommand{\Id}{\mathit{Id}}

\DeclareMathOperator{\Hom}{Hom}
\DeclareMathOperator{\End}{End}
\DeclareMathOperator{\Tr}{Tr}
\DeclareMathOperator{\sgn}{sgn}
\DeclareMathOperator{\rank}{rank}
\DeclareMathOperator{\GL}{GL}

\newcommand{\ang}[1]{\langle #1 \rangle}
\newcommand{\Ang}[1]{\left\langle #1 \right\rangle}
\newcommand{\norm}[1]{\| #1 \|}

\begin{document}

\setcounter{chapter}{5}
\chapter{Burnside's Theorem}

\setcounter{exercise}{1}
\begin{exercise}
  Let $G$ be a non-abelian group of order $39$.
  \begin{enumerate}
    \item Determine the degrees of the irreducible representations of $G$ and how many
    irreducible representations $G$ has of each degree (up to equivalence).
    \begin{answer}
      The degree of an irreducible representation must be $1$ or $3$.
      Let $n$ be the number of degree one representations, $m$ be the number
      of degree $3$ representations, then we have
      $n + 9m = 39$, thus $n \equiv 3 \pmod 9$. Lemma 6.2.7 asserts that
      $n \mid 39$, we must have $n=3$, $m=4$.
    \end{answer}
    \item Determine the number of conjugacy classes of $G$.
    \begin{answer}
      $n + m = 7$.
    \end{answer}
  \end{enumerate}
\end{exercise}

\begin{exercise}
  Let $G$ be a non-abelian group of order $21$.
  \begin{enumerate}
    \item Determine the degrees of the irreducible representations of $G$ and how many
    irreducible representations $G$ has of each degree (up to equivalence).
    \begin{answer}
      The degree of an irreducible representation must be $1$ or $3$.
      Let $n$ be the number of degree one representations, $m$ be the number
      of degree $3$ representations, then we have
      $n + 9m = 21$, thus $n \equiv 3 \pmod 9$. Lemma 6.2.7 asserts that
      $n \mid 21$, we must have $n=3, m = 2$.
    \end{answer}
    \item Determine the number of conjugacy classes of $G$.
    \begin{answer}
      $n + m = 5$.
    \end{answer}
  \end{enumerate}
\end{exercise}

\setcounter{exercise}{4}
\begin{exercise}
  Show that if $\varphi\colon G \to \GL_d(\bbC)$ is a representation with character
  $\chi$, then $g \in \ker \varphi$ if and only if $\chi(g) = d$.
  \begin{proof}
    Since $\varphi_g$ has $d$ eigenvalues that are roots of unity $\lambda_1^n = \dots = \lambda_d^n = 1$,
    we have $\chi(g) = \lambda_1 + \dots + \lambda_d = d$ iff $\lambda_1 = \dots = \lambda_d = 1$,
    i.e., $\varphi_g = I$.
  \end{proof}
\end{exercise}

\begin{exercise}
  For $\alpha\in \bbC$, denote by $\bbZ[\alpha]$ the smallest
  subring of $\bbC$ containing $\alpha$.
  \begin{enumerate}
    \item Prove that $\bbZ[\alpha]=\{a_0 +a_1 \alpha+\dots +a_n \alpha^n
    \mid n\in \bbN, a_0,\dots,a_n \in \bbZ \}$.
    \begin{proof}
      Clearly these elements are closed under addition
      and multiplication, so it is a ring. By the axiom of ring, every $a_0 +a_1 \alpha+\dots +a_n \alpha^n$
      must be in $\bbZ[\alpha]$, so we have the minimality.
    \end{proof}
    \item Prove that the following are equivalent:
    \begin{enumerate}
      \item $\alpha$ is an algebraic integer;
      \item The additive group of $\bbZ[\alpha]$ is finitely generated;
      \item $\alpha$ is contained in a finitely generated subgroup of the additive group of $\bbC$,
      which is closed under multiplication by $\alpha$.
    \end{enumerate}
    \begin{proof}
      (a) implies (b): Suppose $\alpha$ is an algebraic integer, we have
      $a_0 + a_1\alpha + \dots + a_{n-1}\alpha^{n-1} + \alpha^n = 0$.
      Therefore $1, \alpha, \dots, \alpha^{n-1}$ generates $\bbZ[\alpha]$.
      
      (b) clearly implies (c).

      (c) implies (a): Let $y_1, \dots, y_n$ be a set of generators, $\alpha y_i$
      then can be represented by $y_1, \dots, y_n$, by Lemma 6.1.5, $\alpha$ is an
      algebraic integer.
    \end{proof}
  \end{enumerate}
\end{exercise}

\end{document}
