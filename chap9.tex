%!TEX TS-program = xelatex
%!TEX encoding = UTF-8 Unicode
\documentclass[11pt]{report}
\usepackage[margin = 1in]{geometry}

\usepackage{amsmath}
\usepackage{amssymb}
\usepackage{mleftright}
\usepackage{enumitem}
\usepackage{textcomp, gensymb}
\usepackage{tikz, tikz-cd}
\usepackage{bbding}
\usepackage{tabularx}
\usepackage{booktabs}
\usepackage{nicematrix}
\usepackage{extarrows}
\usepackage{lastpage}
\usepackage{hyperref}

\usepackage{fancyhdr}  % Header and Footer formatting

\usetikzlibrary{decorations.markings}

% \usepackage{mathpazo}
\usepackage{unicode-math}
\setmainfont[
	BoldFont={TeX Gyre Termes Bold},
	ItalicFont={TeX Gyre Termes Italic},
	BoldItalicFont={TeX Gyre Termes Bold Italic},
	PunctuationSpace=2
]{TeX Gyre Termes}
\setmathfont
	[Extension = .otf,
	math-style= TeX,
  BoldFont = XITSMath-Bold.otf,
  BoldItalicFont = XITS-BoldItalic.otf
]{XITSMath-Regular}

\usepackage{multicol}

\hypersetup{colorlinks = true,
            linkcolor = blue,
            citecolor = red,
            urlcolor = teal}

\pagestyle{fancy}
\renewcommand{\headrulewidth}{0.4pt}
\renewcommand{\footrulewidth}{0.4pt}
\setlength{\headheight}{18pt}

% Header and Footer Information
\lhead{\small\emph{Baitian Li}}
\chead{}
\rhead{\textsc{Representation Theory of Finite Groups}}
\lfoot{\today}
\cfoot{}
\rfoot{\thepage\ of \pageref{LastPage}}  % Counts the pages.

\makeatletter % This provides a total page count as \ref{NumPages}
\AtEndDocument{\immediate\write\@auxout{\string\newlabel{NumPages}{{\thepage}}}}
\makeatother

%\lineskiplimit=-\maxdimen\relax

\usepackage{amsthm}  % This will create the Problem environment
\newtheorem*{lemma}{Lemma}
\newtheorem*{theorem}{Theorem}
\newtheoremstyle{mythm}%
    {12pt}{}%
    {\sffamily}{}%
    {\bfseries \sffamily}{.}%
    {.5em}%
    {\thmname{#1}\thmnumber{ #2}\thmnote{ #3}}

\theoremstyle{mythm}

\expandafter\let\expandafter\oldproof\csname\string\proof\endcsname
\let\oldendproof\endproof
\renewenvironment{proof}[1][\proofname]{%
  \oldproof[\normalfont \bfseries #1]%
}{\oldendproof}

\newtheorem{exercise}{Exercise}[chapter]
\renewcommand*{\proofname}{Proof}

\newtheoremstyle{myans}%
    {12pt}{}%
    {}{}%
    {\bfseries}{.}%
    {.5em}%
    {\thmname{#1}\thmnumber{ #2}\thmnote{ #3}}

\theoremstyle{myans}
\newtheorem*{answer}{Answer}

\setlist[enumerate]{noitemsep, topsep = 0.2em}
\setlist[enumerate, 1]{label = {\arabic*.}}
% \setlist[enumerate, 2]{label = {(\alph*)}}
\setlist[description]{topsep = 0.2em, listparindent = \parindent, font = \normalfont,
  itemsep = 0em}

\newcommand{\bbR}{\mathbb R}
\newcommand{\bbN}{\mathbb N}
\newcommand{\bbZ}{\mathbb Z}
\newcommand{\bbQ}{\mathbb Q}
\newcommand{\bbC}{\mathbb C}
\newcommand{\Id}{\mathit{Id}}

\DeclareMathOperator{\Hom}{Hom}
\DeclareMathOperator{\End}{End}
\DeclareMathOperator{\Tr}{Tr}
\DeclareMathOperator{\sgn}{sgn}
\DeclareMathOperator{\rank}{rank}
\DeclareMathOperator{\GL}{GL}
\DeclareMathOperator{\Ind}{Ind}
\DeclareMathOperator{\Res}{Res}

\newcommand{\ang}[1]{\langle #1 \rangle}
\newcommand{\Ang}[1]{\left\langle #1 \right\rangle}
\newcommand{\norm}[1]{\| #1 \|}

\begin{document}

\setcounter{chapter}{8}
\chapter{Another Theorem of Burnside}

\begin{exercise}
  Let $G$ be a finite group.
  \begin{enumerate}
    \item Prove that two elements $g, h \in G$ are conjugate if and only
    if $\chi(g) = \chi(h)$ for all irreducible characters $\chi$.
    \begin{proof}
      Suppose $\chi(g) = \chi(h)$ for all characters $\chi$, we have
      $\sum_{\chi \in \mathsf{X}} \chi(g)\overline{\chi(h)} > 0$,
      by the second orthogonality relations, we must have $g$ and $h$ are conjugate.
    \end{proof}
    \item Show that the conjugacy class $C$ of an element $g \in G$ is real if and only if $\chi(g)$
    is real for all irreducible characters $\chi$.
    \begin{proof}
      Let $C$ be the conjugacy class of $g$. $g$ is real iff $g$ and $g^{-1}$ are conjugate
      iff $\chi(g) = \chi(g^{-1}) = \overline{\chi(g)}$ for all $\chi$.
    \end{proof}
  \end{enumerate}
\end{exercise}

\begin{exercise}
  Prove that all characters of the symmetric group $S_n$ are real.
  \begin{proof}
    For every $\sigma\in S_n$, since the cycles of $\sigma$ and $\sigma^{-1}$ are the same
    integer partition, they are conjugate. So all conjugacy class of $S_n$ are real, thus all
    characters of $S_n$ are real.
  \end{proof}
\end{exercise}

\begin{exercise}
  Let $G$ be a non-abelian group of order $155$. Prove that $G$ has $11$ conjugacy classes.
  \begin{proof}
    Noticed that $155 = 5\cdot 31$, the degree of irreducible presentations of $G$ can only be
    $1$ or $5$. The number of degree one representations has the form $2n+1$,
    the number of degree $5$ representations has the form $2m + 2$. So we have
    $(2n+1) + (2m+2)\cdot 5^2 = 155$. We have $2n+50m = 104$. Noticed that the number of
    degree one representations is $|G/G'| \mid |G|$, we have $2n+1 \mid 155$, with these restrictions
    we have the only solution $(n, m) = (2, 2)$. So $(2n+1) + (2m+2) = 11$.
  \end{proof}
\end{exercise}

\begin{exercise}
  Let $G$ be a group and let $\alpha$ be an automorphism of $G$.
  \begin{enumerate}
    \item If $\varphi\colon G \to \GL(V)$ is a representation, show that
    $\alpha^*(\varphi) =\varphi \circ \alpha$ is a
    representation of $G$.
    \begin{proof}
      Trivial.
    \end{proof}
    \item If $f \in L(G)$, define $\alpha^*(f) = f \circ \alpha$. Prove that
    $\chi_{\alpha^*(\varphi)} = \alpha^*(\chi_\varphi)$.
    \begin{proof}
      $\chi_{\alpha^*(\varphi)}(g) = \Tr [\alpha^*(\varphi)(g)]
      = \Tr [\varphi(\alpha(g))] = \chi_\varphi(\alpha(g)) = \alpha^*(\chi_\varphi)(g)$.
    \end{proof}
    \item Prove that if $\varphi$ is irreducible, then so is $\alpha^*(\varphi)$.
    \begin{proof}
      $\ang{\alpha^*(\chi), \alpha^*(\chi)} = \frac 1{|G|}\sum_{g\in G} |\chi(\alpha(g))|
      = \frac 1{|G|}\sum_{g\in G} |\chi(g)| = \ang{\chi, \chi} = 1$.
    \end{proof}
    \item Show that if $C$ is a conjugacy class of $G$, then $\alpha(C)$ is also a conjugacy class
    of $G$.
    \begin{proof}
      $y = gxg^{-1} \iff \alpha(y) = \alpha(g) \alpha(x) \alpha(g)^{-1}$.
    \end{proof}
    \item Prove that the number of conjugacy classes $C$ with $\alpha(C) = C$ is equal to the
    number of irreducible characters $\chi$ with $\alpha^*(\chi) = \chi$.
    \begin{proof}
      Let $s$ be the number of conjugacy classes. Let $\varphi\colon S_s \to \GL_s(\bbC)$ be the
      standard representation.
      Consider $\varphi_{\alpha^*} \mathsf{X}$, it permutes the row $\chi$ to the row $\alpha^*(\chi)$,
      we have $(\varphi_{\alpha^*}\mathsf X)_{\chi C} = (\alpha^*)^{-1}(\chi)(C)
      = \chi(\alpha^{-1}(C)) = (\mathsf X \varphi_{\alpha})_{\chi C}$, so
      $\varphi_\alpha = \mathsf X^{-1} \varphi_{\alpha*} \mathsf X$. Thus
      $\Tr \varphi_\alpha = \Tr \varphi_{\alpha^*}$, they fixes the same number of elements.
    \end{proof}
  \end{enumerate}
\end{exercise}

\end{document}
