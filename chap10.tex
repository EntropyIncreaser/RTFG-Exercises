%!TEX TS-program = xelatex
%!TEX encoding = UTF-8 Unicode
\documentclass[11pt]{report}
\usepackage[margin = 1in]{geometry}

\usepackage{amsmath}
\usepackage{amssymb}
\usepackage{mleftright}
\usepackage{enumitem}
\usepackage{textcomp, gensymb}
\usepackage{tikz, tikz-cd}
\usepackage{bbding}
\usepackage{tabularx}
\usepackage{booktabs}
\usepackage{nicematrix}
\usepackage{extarrows}
\usepackage{lastpage}
\usepackage{hyperref}

\usepackage{fancyhdr}  % Header and Footer formatting

\usetikzlibrary{decorations.markings}

% \usepackage{mathpazo}
\usepackage{unicode-math}
\setmainfont[
	BoldFont={TeX Gyre Termes Bold},
	ItalicFont={TeX Gyre Termes Italic},
	BoldItalicFont={TeX Gyre Termes Bold Italic},
	PunctuationSpace=2
]{TeX Gyre Termes}
\setmathfont
	[Extension = .otf,
	math-style= TeX,
  BoldFont = XITSMath-Bold.otf,
  BoldItalicFont = XITS-BoldItalic.otf
]{XITSMath-Regular}

\usepackage{multicol}

\hypersetup{colorlinks = true,
            linkcolor = blue,
            citecolor = red,
            urlcolor = teal}

\pagestyle{fancy}
\renewcommand{\headrulewidth}{0.4pt}
\renewcommand{\footrulewidth}{0.4pt}
\setlength{\headheight}{18pt}

% Header and Footer Information
\lhead{\small\emph{Baitian Li}}
\chead{}
\rhead{\textsc{Representation Theory of Finite Groups}}
\lfoot{\today}
\cfoot{}
\rfoot{\thepage\ of \pageref{LastPage}}  % Counts the pages.

\makeatletter % This provides a total page count as \ref{NumPages}
\AtEndDocument{\immediate\write\@auxout{\string\newlabel{NumPages}{{\thepage}}}}
\makeatother

%\lineskiplimit=-\maxdimen\relax

\usepackage{amsthm}  % This will create the Problem environment
\newtheorem*{lemma}{Lemma}
\newtheorem*{theorem}{Theorem}
\newtheorem*{corollary}{Corollary}
\newtheoremstyle{mythm}%
    {12pt}{}%
    {\sffamily}{}%
    {\bfseries \sffamily}{.}%
    {.5em}%
    {\thmname{#1}\thmnumber{ #2}\thmnote{ #3}}

\theoremstyle{mythm}

\expandafter\let\expandafter\oldproof\csname\string\proof\endcsname
\let\oldendproof\endproof
\renewenvironment{proof}[1][\proofname]{%
  \oldproof[\normalfont \bfseries #1]%
}{\oldendproof}

\newtheorem{exercise}{Exercise}[chapter]
\renewcommand*{\proofname}{Proof}

\newtheoremstyle{myans}%
    {12pt}{}%
    {}{}%
    {\bfseries}{.}%
    {.5em}%
    {\thmname{#1}\thmnumber{ #2}\thmnote{ #3}}

\theoremstyle{myans}
\newtheorem*{answer}{Answer}

\setlist[enumerate]{noitemsep, topsep = 0.2em}
\setlist[enumerate, 1]{label = {\arabic*.}}
% \setlist[enumerate, 2]{label = {(\alph*)}}
\setlist[description]{topsep = 0.2em, listparindent = \parindent, font = \normalfont,
  itemsep = 0em}

\newcommand{\bbR}{\mathbb R}
\newcommand{\bbN}{\mathbb N}
\newcommand{\bbZ}{\mathbb Z}
\newcommand{\bbQ}{\mathbb Q}
\newcommand{\bbC}{\mathbb C}
\newcommand{\Id}{\mathit{Id}}

\DeclareMathOperator{\Hom}{Hom}
\DeclareMathOperator{\End}{End}
\DeclareMathOperator{\Tr}{Tr}
\DeclareMathOperator{\sgn}{sgn}
\DeclareMathOperator{\rank}{rank}
\DeclareMathOperator{\GL}{GL}
\DeclareMathOperator{\Ind}{Ind}
\DeclareMathOperator{\Res}{Res}

\newcommand{\ang}[1]{\langle #1 \rangle}
\newcommand{\Ang}[1]{\left\langle #1 \right\rangle}
\newcommand{\norm}[1]{\| #1 \|}

\begin{document}

\setcounter{chapter}{9}
\chapter{Representation Theory of the Symmetric Group}

\setcounter{exercise}{2}
\begin{exercise}
  Prove that if $\lambda = (n - 1, 1)$, then the corresponding Sprecht
  representation of $S_n$ is equivalent to the augmentation subrepresentation of the
  standard representation of $S_n$.
  \begin{proof}
    For $\lambda = (n-1, 1)$, clearly $\varphi^\lambda$ is equivalent to the standard representation.
    For each $\lambda$-tableaux $t$, we have $C_t = S_{\{t_{11}, t_{21}\}}$. Thus
    $A_t = \varphi_1^\lambda - \varphi_{(t_{11}\ t_{21})}^\lambda$, then
    $e_t = [t] - [(t_{11}\ t_{21})t]$. So the spanned subspace $S^\lambda$ is spanned by $x_1 - x_k$ for all $k$,
    that is $\sum_i a_i x_i$ such that $\sum_i a_i = 0$. Let $v_0 = \sum_i x_i$, we have
    $S^\lambda = (\bbC v_0)^\perp$, so $\psi^\lambda$ is the augmentation representation.
  \end{proof}
\end{exercise}

\setcounter{exercise}{5}
\begin{exercise}
  Prove Corollary 10.2.18.
  \begin{corollary}
    Suppose $μ \vdash n$. Then $\psi^\mu$ appears with multiplicity one as an irreducible constituent
    of $\varphi^\mu$. Any other irreducible constituent $\psi^\lambda$ of $\varphi^\mu$
    satisfies $\lambda \trianglerighteq \mu$.
  \end{corollary}
  \begin{proof}
    Suppose the multiplicity is $k$, since $\Hom(\psi^\mu, \varphi \oplus \varphi')
    = \Hom(\psi^\mu, \varphi) \oplus \Hom(\psi^\mu, \varphi')$, we should have
    $\dim \Hom_{S_n}(\psi^\mu, \varphi^\mu) = k$, therefore the multiplicity should be $k=1$.

    Suppose we have such constituent $\psi^\lambda$, we can construct a nonzero homomorphism
    $I_{S^\lambda} \oplus 0_{(S^\lambda)^\perp}$, so $\Hom(\psi^\lambda, \varphi^\mu)$ is nonzero,
    we have $\lambda \trianglerighteq \mu$.
  \end{proof}
\end{exercise}

\begin{exercise}
  Prove that each character of $S_n$ takes on only rational values.
  \begin{proof}
    Let the $\bbQ$-span of $[t]$ be $\bbQ T^\lambda$, then by definition we have
    $e_t \in \bbQ T^\lambda$ for each $\lambda$-tableaux $t$, and each $\psi^\lambda_\sigma$
    has rational entries in the basis $[t]$. So choose a basis $B$ in $\bbQ T^\lambda$,
    $\psi^\lambda_\sigma |_{B}$ still has rational entries, thus
    $\chi^\lambda(g) = \Tr \left[\psi^\lambda_\sigma |_{B}(g)\right] \in \bbQ$.
  \end{proof}
\end{exercise}

\end{document}
