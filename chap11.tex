%!TEX TS-program = xelatex
%!TEX encoding = UTF-8 Unicode
\documentclass[11pt]{report}
\usepackage[margin = 1in]{geometry}

\usepackage{amsmath}
\usepackage{amssymb}
\usepackage{mleftright}
\usepackage{enumitem}
\usepackage{textcomp, gensymb}
\usepackage{tikz, tikz-cd}
\usepackage{bbding}
\usepackage{tabularx}
\usepackage{booktabs}
\usepackage{nicematrix}
\usepackage{extarrows}
\usepackage{lastpage}
\usepackage{hyperref}

\usepackage{fancyhdr}  % Header and Footer formatting

\usetikzlibrary{decorations.markings}

% \usepackage{mathpazo}
\usepackage{unicode-math}
\setmainfont[
	BoldFont={TeX Gyre Termes Bold},
	ItalicFont={TeX Gyre Termes Italic},
	BoldItalicFont={TeX Gyre Termes Bold Italic},
	PunctuationSpace=2
]{TeX Gyre Termes}
\setmathfont
	[Extension = .otf,
	math-style= TeX,
  BoldFont = XITSMath-Bold.otf,
  BoldItalicFont = XITS-BoldItalic.otf
]{XITSMath-Regular}

\usepackage{multicol}

\hypersetup{colorlinks = true,
            linkcolor = blue,
            citecolor = red,
            urlcolor = teal}

\pagestyle{fancy}
\renewcommand{\headrulewidth}{0.4pt}
\renewcommand{\footrulewidth}{0.4pt}
\setlength{\headheight}{18pt}

% Header and Footer Information
\lhead{\small\emph{Baitian Li}}
\chead{}
\rhead{\textsc{Representation Theory of Finite Groups}}
\lfoot{\today}
\cfoot{}
\rfoot{\thepage\ of \pageref{LastPage}}  % Counts the pages.

\makeatletter % This provides a total page count as \ref{NumPages}
\AtEndDocument{\immediate\write\@auxout{\string\newlabel{NumPages}{{\thepage}}}}
\makeatother

%\lineskiplimit=-\maxdimen\relax

\usepackage{amsthm}  % This will create the Problem environment
\newtheorem*{lemma}{Lemma}
\newtheorem*{theorem}{Theorem}
\newtheorem*{corollary}{Corollary}
\newtheoremstyle{mythm}%
    {12pt}{}%
    {\sffamily}{}%
    {\bfseries \sffamily}{.}%
    {.5em}%
    {\thmname{#1}\thmnumber{ #2}\thmnote{ #3}}

\theoremstyle{mythm}

\expandafter\let\expandafter\oldproof\csname\string\proof\endcsname
\let\oldendproof\endproof
\renewenvironment{proof}[1][\proofname]{%
  \oldproof[\normalfont \bfseries #1]%
}{\oldendproof}

\newtheorem{exercise}{Exercise}[chapter]
\renewcommand*{\proofname}{Proof}

\newtheoremstyle{myans}%
    {12pt}{}%
    {}{}%
    {\bfseries}{.}%
    {.5em}%
    {\thmname{#1}\thmnumber{ #2}\thmnote{ #3}}

\theoremstyle{myans}
\newtheorem*{answer}{Answer}

\setlist[enumerate]{noitemsep, topsep = 0.2em}
\setlist[enumerate, 1]{label = {\arabic*.}}
% \setlist[enumerate, 2]{label = {(\alph*)}}
\setlist[description]{topsep = 0.2em, listparindent = \parindent, font = \normalfont,
  itemsep = 0em}

\newcommand{\bbR}{\mathbb R}
\newcommand{\bbN}{\mathbb N}
\newcommand{\bbZ}{\mathbb Z}
\newcommand{\bbQ}{\mathbb Q}
\newcommand{\bbC}{\mathbb C}
\newcommand{\Id}{\mathit{Id}}

\DeclareMathOperator{\Hom}{Hom}
\DeclareMathOperator{\End}{End}
\DeclareMathOperator{\Tr}{Tr}
\DeclareMathOperator{\sgn}{sgn}
\DeclareMathOperator{\rank}{rank}
\DeclareMathOperator{\GL}{GL}
\DeclareMathOperator{\Ind}{Ind}
\DeclareMathOperator{\Res}{Res}
\DeclareMathOperator{\Supp}{Supp}

\newcommand{\ang}[1]{\langle #1 \rangle}
\newcommand{\Ang}[1]{\left\langle #1 \right\rangle}
\newcommand{\norm}[1]{\| #1 \|}

\begin{document}

\setcounter{chapter}{10}
\chapter{Probability and Random Walks on Groups}

\begin{exercise}
  Show that if $P$ is any probability on a finite group $G$, then $P*U=
  U = U * P$. Deduce that $P^{*n} - U = (P - U)^{*n}$.
  \begin{proof}
    We have
    \[ (P*U)(g) = \sum_{h\in G} P(gh^{-1})U(h)
    = \frac 1{|G|} \sum_{h\in G} P(gh^{-1}) = \frac 1{|G|} = U(g). \]
    Similarly,
    \[ (U*P)(g) = \sum_{h\in G} U(gh^{-1})P(h)
    = \frac 1{|G|} \sum_{h\in G} P(h) = \frac 1{|G|} = U(g). \]
    Thus by induction, we have
    \[ (P-U)^{*n} = (P^{*(n-1)} - U)*(P-U)
    = P^{*n} - U - P^{*(n-1)}*U + U
    = P^{*n} - U, \]
    here $P^{*n}*U = U$ is also by induction.
  \end{proof}
\end{exercise}

\begin{exercise}
  Show that if $P$ is a probability on a finite group $G$ such that $P =
  P * P$, then there is a subgroup $H$ of $G$ such that $P$ is the uniform
  distribution on $H$.
  \begin{proof}
    Noticed that $\Supp P = \Supp(P*P) = \Supp P \cdot \Supp P$,
    we have $H=\Supp P$ is nonempty and closed under multiplication.
    Since $G$ is finite, $H$ is a subgroup. Therefore, we only need to prove
    that if $\Supp P = G$ and $P*P = P$, then $P = U$.

    Since $\Supp P = G$, we have $P$ is ergodic, thus
    $\lim_{n\to \infty} \norm{P^{*n} - U}_{TV} = 0$.
    Sicne $P*P=P$, we have $P^{*n} = P$, so
    $\norm{P-U} = 0$, we have $P = U$.
  \end{proof}
\end{exercise}

\setcounter{exercise}{7}
\begin{exercise}
  Let $P$ be a probability on a finite group $G$.
  \begin{enumerate}[label=(\alph*)]
    \item Show that if the sequence $(P^{*k} )$ converges to $U$, then $P$ is ergodic.
    \begin{proof}
      By definition, we have $N_g$ such that $P^{*n}(g) > 0$ for each $n \geq N_g$ for each $g$.
      Then take $N = \max_{g\in G} N_g$, we have $\Supp(P^{*N}) = G$, thus $P$ is ergodic.
    \end{proof}
    \item Suppose in addition that $G$ is abelian. Show that the random walk driven by $P$
    is ergodic if and only if $|\hat P(\chi)| < 1$ for all $\chi \in \hat G^{*}$. 
    \begin{proof}
      If $|\hat P(\chi)| < 1$ for all $\chi \in \hat G^{*}$, then by the Upper bound lemma,
      we have $\norm{P^{*n}-U}_{TV}$ converges to $0$, thus $P^{*n}$ converges to $U$,
      thus $P$ is ergodic. Conversely, if $P$ is ergodic, we have $\Supp P^{*N} = G$ for some $N$.
      Since the equation of $|\hat P(\chi)| = |\sum_{g\in G} P(g)\chi(g)| \leq 1$ holds only if
      all $P(g)\chi(g)$ are on the same direction, it cannot be true for any nontrivial
      character.
    \end{proof}
  \end{enumerate}
\end{exercise}

\end{document}
