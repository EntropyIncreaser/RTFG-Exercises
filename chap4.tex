%!TEX TS-program = xelatex
%!TEX encoding = UTF-8 Unicode
\documentclass[11pt]{report}
\usepackage[margin = 1in]{geometry}

\usepackage{amsmath}
\usepackage{amssymb}
\usepackage{mleftright}
\usepackage{enumitem}
\usepackage{textcomp, gensymb}
\usepackage{tikz, tikz-cd}
\usepackage{bbding}
\usepackage{tabularx}
\usepackage{nicematrix}
\usepackage{extarrows}
\usepackage{lastpage}
\usepackage{hyperref}

\usepackage{fancyhdr}  % Header and Footer formatting

\usetikzlibrary{decorations.markings}

% \usepackage{mathpazo}
\usepackage{unicode-math}
\setmainfont[
	BoldFont={TeX Gyre Termes Bold},
	ItalicFont={TeX Gyre Termes Italic},
	BoldItalicFont={TeX Gyre Termes Bold Italic},
	PunctuationSpace=2
]{TeX Gyre Termes}
\setmathfont
	[Extension = .otf,
	math-style= TeX,
  BoldFont = XITSMath-Bold.otf,
  BoldItalicFont = XITS-BoldItalic.otf
]{XITSMath-Regular}

\usepackage{multicol}

\hypersetup{colorlinks = true,
            linkcolor = blue,
            citecolor = red,
            urlcolor = teal}

\pagestyle{fancy}
\renewcommand{\headrulewidth}{0.4pt}
\renewcommand{\footrulewidth}{0.4pt}
\setlength{\headheight}{18pt}

% Header and Footer Information
\lhead{\small\emph{Baitian Li}}
\chead{}
\rhead{\textsc{Representation Theory of Finite Groups}}
\lfoot{\today}
\cfoot{}
\rfoot{\thepage\ of \pageref{LastPage}}  % Counts the pages.

\makeatletter % This provides a total page count as \ref{NumPages}
\AtEndDocument{\immediate\write\@auxout{\string\newlabel{NumPages}{{\thepage}}}}
\makeatother

%\lineskiplimit=-\maxdimen\relax

\usepackage{amsthm}  % This will create the Problem environment
\newtheorem*{lemma}{Lemma}
\newtheorem*{theorem}{Theorem}
\newtheoremstyle{mythm}%
    {12pt}{}%
    {\sffamily}{}%
    {\bfseries \sffamily}{.}%
    {.5em}%
    {\thmname{#1}\thmnumber{ #2}\thmnote{ #3}}

\theoremstyle{mythm}

\expandafter\let\expandafter\oldproof\csname\string\proof\endcsname
\let\oldendproof\endproof
\renewenvironment{proof}[1][\proofname]{%
  \oldproof[\normalfont \bfseries #1]%
}{\oldendproof}

\newtheorem{exercise}{Exercise}[chapter]
\renewcommand*{\proofname}{Proof}

\newtheoremstyle{myans}%
    {12pt}{}%
    {}{}%
    {\bfseries}{.}%
    {.5em}%
    {\thmname{#1}\thmnumber{ #2}\thmnote{ #3}}

\theoremstyle{myans}
\newtheorem*{answer}{Answer}

\setlist[enumerate]{noitemsep, topsep = 0.2em}
\setlist[enumerate, 1]{label = {\arabic*.}}
% \setlist[enumerate, 2]{label = {(\alph*)}}
\setlist[description]{topsep = 0.2em, listparindent = \parindent, font = \normalfont,
  itemsep = 0em}

\newcommand{\bbR}{\mathbb R}
\newcommand{\bbN}{\mathbb N}
\newcommand{\bbZ}{\mathbb Z}
\newcommand{\bbQ}{\mathbb Q}
\newcommand{\bbC}{\mathbb C}
\newcommand{\Id}{\mathit{Id}}

\DeclareMathOperator{\Hom}{Hom}
\DeclareMathOperator{\End}{End}
\DeclareMathOperator{\Tr}{Tr}
\DeclareMathOperator{\sgn}{sgn}
\DeclareMathOperator{\rank}{rank}
\DeclareMathOperator{\GL}{GL}

\newcommand{\ang}[1]{\langle #1 \rangle}
\newcommand{\Ang}[1]{\left\langle #1 \right\rangle}
\newcommand{\norm}[1]{\| #1 \|}

\begin{document}

\setcounter{chapter}{3}
\chapter{Character Theory and the Orthogonality Relations}

\begin{exercise}
  Let $\varphi\colon G \to \GL(U)$, $\psi\colon G \to \GL(V )$ and
  $\rho\colon G \to \GL(W )$ be representations of a group $G$. Suppose that
  $T\in \Hom_G(\varphi, \psi)$ and $S\in \Hom_G(\psi, \rho)$.
  Prove that $ST \in \Hom_G(\psi, \rho)$.
  \begin{proof}
    For each $g$, $ST \varphi_g = S \psi_g T = \rho_g ST$, so $ST\in \Hom_G(\psi, \rho)$.
  \end{proof}
\end{exercise}

\begin{exercise}
  Let $\varphi$ be a representation of a group $G$ with character $\chi_\varphi$. Prove that
  $\chi_\varphi(g^{-1}) =\overline{\chi_\varphi(g)}$.
  \begin{proof}
    $\varphi$ is equivalent to some unitary representation $\rho$, then we have $\chi_\varphi = \chi_\rho$.
    Then we have $\chi_\rho(g^{-1}) = \Tr(\rho_{g^{-1}}) = \Tr(\rho_g^*) = \overline{\Tr \rho_g} = \overline{\chi_\rho(g)}$.
  \end{proof}
\end{exercise}

\begin{exercise}
  Let $\varphi\colon G \to \GL(V )$ be an irreducible representation. Let
  \[ Z(G) = \{ a\in G \mid ag = ga, \forall g\in G\} \]
  be the center of $G$. Show that if $a \in Z(G)$, then $\varphi(a) = \lambda I$ for some $\lambda \in \bbC^*$.
  \begin{proof}
    Since $a\in Z(G)$, we have $\varphi_a \varphi_g = \varphi_g \varphi_a$ for all $g$, so
    $\varphi_a\in \Hom_G(\varphi, \varphi)$. Then we have $\varphi_a = \lambda I$, since
    $\varphi_a\in \GL(V)$, we have $\lambda \neq 0$.
  \end{proof}
\end{exercise}

\begin{exercise}
  Let $G$ be a group. Show that $f \colon G \to \bbC$ is a class function if and
  only if $f(gh) = f(hg)$ for all $g, h \in G$.
  \begin{proof}
    Since this is for all $g, h$, we substitude by $g = h^{-1}x$, the statement
    is equivalent to $f(h^{-1}xh) = f(x)$ for all $x, h\in G$. Which is the definition
    of class function.
  \end{proof}
\end{exercise}

\begin{exercise}
  For $v = (c_1, \dots , c_m) \in (\bbZ/2\bbZ)^m$, let $\alpha(v) = \{i \mid c_i = [1]\}$. For
  each $Y \subseteq \{1, \dots, m\}$, define a function $\chi_Y\colon (\bbZ/2\bbZ)^m \to \bbC$ by
  \[ \chi_Y (v) = (-1)^{|\alpha(v) \cap Y|}. \]
  \begin{enumerate}
    \item Show that $\chi_Y$ is a character.
    \begin{proof}
      We only need to prove that $\chi_Y$ is a representation, this can be verified by definition.
    \end{proof}
    \item Show that every irreducible character of $(\bbZ/2\bbZ)^m$ is of the form $\chi_Y$ for some
    subset $Y$ of $\{1, \dots , m\}$.
    \begin{proof}
      Since $(\bbZ/2\bbZ)^m$ is abelian, the irreducible representation must have degree $1$.
      Thus such character $\chi$ is a representation. Let $Y = \{ i \mid e_i = ([0], \dots, [1], \dots, [0]) = -1 \}$,
      where the $i$th position of $e_i$ is $[1]$.
      Then $\chi$ agrees with $\chi_Y$ on $\{ e_i \}$, which generates the whole $(\bbZ/2\bbZ)^m$.
      Then we have $\chi = \chi_Y$.
    \end{proof}
    \item Show that if $X, Y \subseteq \{1, \dots , m\}$, then $\chi_X(v)\chi_Y (v) = \chi_{X\Delta Y}(v)$ where
    $X \Delta Y = X\cup Y \smallsetminus (X\cap Y)$ is the symmetric difference.
    \begin{proof}
      Can be verified by definition.
    \end{proof}
  \end{enumerate}
\end{exercise}

\begin{exercise}
  Let $\sgn\colon S_n \to \bbC^*$ be the representation given by
  \[ \sgn(\sigma) = \begin{cases}
    1 & \sigma \text{ is even}\\
    -1 & \sigma \text{ is odd.}
  \end{cases} \]
  Show that if $\chi$ is the character of an irreducible representation of $S_n$ not equivalent
  to $\sgn$, then
  \[ \sum_{\sigma \in S_n} \sgn(\sigma) \chi(\sigma) = 0. \]
  \begin{proof}
    Since $\sgn$ has degree $1$, it is irreducible and $\chi_{\sgn} = \sgn$.
    The equation directly follows from $\ang{\chi, \chi_{\sgn}} = 0$, which is implied by
    the first orthogonality relations.
  \end{proof}
\end{exercise}

\begin{exercise}
  Let $\varphi\colon G \to \GL_n(\bbC)$ and $\rho\colon G \to \GL_m(\bbC)$ be representations.
  Let $V = M_{mn}(\bbC)$. Define $\tau\colon G \to \GL(V )$ by $\tau_g(A) = \rho_g A \varphi^T_g$.
  \begin{enumerate}
    \item Show that $\tau$ is a representation of $G$.
    \begin{proof}
      Clearly $\tau_g$ is a linear mapping, and $\tau_1(A) = \rho_1 A \varphi_1^T = A$, and
      $\tau_g \tau_h(A) = \tau_g(\rho_h A \varphi_h^T) = \rho_{gh}A\varphi_{gh}^T = \tau_{gh}(A)$.
      Thus $\tau_g \in \GL(V)$ for all $g$, and $\tau$ is a homomorphism, therefore $\tau$ is a representation.
    \end{proof}
    \item Show that
    \[ \tau_g E_{kl} = \sum_{i,j} \rho_{ik}(g) \varphi_{jl}(g) E_{ij}. \]
    \begin{proof}
      By definition.
    \end{proof}
    \item Prove that $\chi_\tau(g) = \chi_\rho(g)\chi_\varphi(g)$.
    \begin{proof}
      \[ \chi_\tau(g) = \Tr(\tau_g) = \sum_{i, j} (\tau_g E_{ij})_{ij}
      =\sum_{i,j} \rho_{ii}(g) \varphi_{jj}(g) = \chi_\rho(g) \chi_\varphi(g). \qedhere \]
    \end{proof}
    \item Conclude that the pointwise product of two characters of $G$ is a character of $G$.
    \begin{proof}
      The product character $\chi_\rho(g) \chi_\varphi(g)$ is given by the representation $\tau$.
    \end{proof}
  \end{enumerate}
\end{exercise}

\begin{exercise}
  Let $\alpha\colon S_n \to \GL_n(\bbC)$ be the standard representation from
  Example 3.1.9.
  \begin{enumerate}
    \item Show that $\chi_\alpha(\sigma)$ is the number of fixed points of $\sigma$, that is, the number of
    elements $k \in \{1, \dots , n\}$ such that $\sigma(k) = k$.
    \begin{proof}
      \[ \chi_\alpha(\sigma) = \Tr (\alpha(\sigma)) = \sum_{i} \ang{e_i, e_{\sigma(i)}}. \qedhere \]
    \end{proof}
    \item Show that if $n = 3$, then $\ang{\chi_\alpha, \chi_\alpha} = 2$ and hence $\alpha$ is not irreducible.
    \begin{proof}
      We have $\ang{\chi_\alpha, \chi_\alpha} = \frac 1{6} \sum_{\sigma} \|\chi_\alpha(\sigma)\|^2$.
      \begin{itemize}
        \item For $\sigma = 1$, $\norm{\chi_\alpha(\sigma)}^2 = 9$.
        \item For $\sigma = (1\ 2), (2\ 3), (3\ 1)$, $\norm{\chi_\alpha(\sigma)}^2 = 1$.
        \item Otherwise, $\norm{\chi_\alpha(\sigma)}^2 = 0$.
      \end{itemize}
      So the sum is $9 + 3\cdot 1 = 12$, we have $\ang{\chi_\alpha, \chi_\alpha} = 2$.
    \end{proof}
  \end{enumerate}
\end{exercise}

\begin{exercise}
  Let $\chi$ be a non-trivial irreducible character of a finite group $G$. Show
  that
  \[ \sum_{g\in G} \chi(g) = 0. \]
  \begin{proof}
    Since the trivial representation $1$ is irreducible, we have a irreducible character $\chi_1(g) = 1$
    for all $g$. Since $\chi\neq \chi_1$, by the first orthogonality relations we have
    $\ang{\chi, \chi_1} = 0$.
  \end{proof}
\end{exercise}

\begin{exercise}
  Let $\varphi\colon G\to H$ be a surjective homomorphism and let
  $\psi \colon H\to \GL(V )$ be an irreducible representation. Prove that $\psi \circ \varphi$ is an irreducible
  representation of $G$.
  \begin{proof}
    Suppose $W\leq V$ is $G$-invariant, then for each $h\in H$, since $\varphi$ is surjective, we have
    $\varphi(g) =h$, thus $\psi_h W = (\psi \circ \varphi)_g W \leq W$. We must have $W = 0$ or $W = V$.
  \end{proof}
\end{exercise}

\setcounter{exercise}{12}
\begin{exercise}
  Let $G$ be a group and let $G'$ be the commutator subgroup of $G$. Let $\varphi: G \to G/G'$ be the
  canonical homomorphism given by $\varphi(g) = gG'$. Prove
  that every degree one representation $\rho\colon G \to \bbC^*$ is of the form $\psi \circ \varphi$ where
  $\psi\colon G/G' \to \bbC^*$ is a degree one representation of the abelian group $G/G'$.
  \begin{proof}
    Let $N = \ker \rho$, since $G/N \cong \rho(G) \subset \bbC^*$ is commutative, we have
    $G' \leq N$. Then $\rho$ induces the homomorphism $\psi$ such that $\rho = \psi \circ \varphi$
    naturally.
  \end{proof}
\end{exercise}

\begin{exercise}
  Show that if $G$ is a finite group and $g$ is a non-trivial element
  of $G$, then there is an irreducible representation $ϕ$ with $\varphi(g) \neq I$.
  \begin{proof}
    Consider the regular representation $L\colon G \to \GL(\bbC G)$, since $L_g e = g$,
    we have $L_g \neq I$. Consider the decomposition of representation
    $L \sim d_1\varphi^{(1)} \oplus \dots \oplus d_s\varphi^{(s)}$, suppose $\varphi^{(i)}_g$
    is identity for all $i$, we must have $L_g= I$, so we must have some $\varphi^{(i)}$
    such that $\varphi^{(i)}_g \neq I$.
  \end{proof}
\end{exercise}

\begin{exercise}
  This exercise provides an alternate proof of the orthogonality
  relations for characters. It requires Exercise 3.5. Let $\varphi\colon G \to \GL_m(\bbC)$ and
  $\psi\colon G \to \GL_n(\bbC)$ be representations of a finite group $G$.
  Let $V = M_{mn}(\bbC)$ and define a representation $\rho\colon G \to \GL(V )$ by $\rho_g(A) = \varphi_g A \psi_{g}^*$.
  \begin{enumerate}
    \item Prove that $\chi_\rho(g) = \chi_\varphi(g)\overline{\chi_\psi(g)}$ for $g \in G$.
    \begin{proof}
      Follows from Exercise 4.7.
    \end{proof}
    \item Show that $V^G = \Hom_G(\psi, \varphi)$.
    \begin{proof}
      $A \in V^G \iff \varphi_g A \psi_g^* = A \iff \varphi_g A = A \psi_g \iff A \in \Hom_G(\psi, \varphi)$.
    \end{proof}
    \item Deduce $\dim \Hom_G(\psi, \varphi) = \ang{\chi_\varphi, \chi_\psi}$ using Exercise 3.5.
    \begin{proof}
      \[ \ang{\chi_\varphi, \chi_\psi} = \frac 1{|G|} \sum_{g\in G} \chi_\varphi(g)\overline{\chi_\psi(g)}
      = \frac 1{|G|} \sum_{g\in G} \chi_\rho(g)
      \xlongequal{\text{exercise 3.5}} \dim V^G = \dim \Hom_G(\psi, \varphi). \qedhere \]
    \end{proof}
    \item Deduce using Schur's lemma that if $\varphi$ and $\psi$ are irreducible representations, then
    \[ \ang{\chi_\varphi, \chi_\psi} = \begin{cases}
      1 & \varphi \sim \psi,\\
      0 & \varphi \nsim \psi.
    \end{cases} \]
    \begin{proof}
      Schur's lemma states that,
      if $\varphi \nsim \psi$, we have $\Hom_G(\psi, \varphi) = 0$, otherwise
      $\Hom_G(\psi, \varphi) = \{ \lambda I \mid \lambda \in \bbC \}$.
    \end{proof}
  \end{enumerate}
\end{exercise}

\end{document}
